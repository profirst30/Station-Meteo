\documentclass[12pt]{article}
%\usepackage[a4paper,width=150mm,top=10mm,bottom=20mm]{geometry}
\usepackage[a4paper,width=160mm,top=25mm,bottom=25mm]{geometry}



%%%%% For Title Page %%%%%
\usepackage{graphicx}
\usepackage{subcaption}
\usepackage{etoolbox}
\usepackage{relsize}
%%%%%%%%%%%%%%%%%%%%%%%%%%

\usepackage[utf8]{inputenc}





% Redéfinition des marques pour utiliser la section et la sous-section
\renewcommand{\sectionmark}[1]{\markboth{#1}{}}
\renewcommand{\subsectionmark}[1]{\markright{\thesubsection\ #1}}



\newcommand{\nsection}[1]{%
  \section*{#1}%
  \addcontentsline{toc}{section}{#1}%
  \markboth{#1}{}% Met à jour les deux marques pour les sections non numérotées
  
}

\newcommand{\nsubsection}[1]{%
  \subsection*{#1}%
  \addcontentsline{toc}{subsection}{#1}%
  \markright{#1}% Met à jour la marque de droite pour les sous-sections non numérotées
}




\usepackage[utf8]{inputenc}
\usepackage[T1]{fontenc}
\usepackage[french]{babel}
\usepackage{titling}
\usepackage{graphicx}
\usepackage{svg}
% Ajouter dans le préambule du document :
\usepackage{minted}
\usepackage{xcolor}
\usepackage{float}
\floatplacement{figure}{H}
\usepackage[colorlinks=true, linkcolor=blue]{hyperref}
\usepackage[all]{hypcap}
\usepackage{amsmath}
\usepackage{fancyhdr}
\usepackage{lastpage}



%\geometry{margin=2.5cm}
\setlength{\headheight}{15pt}

% Configuration de fancyhdr

\fancyhf{} % Efface tous les champs d'en-tête et de pied de page
\renewcommand{\headrulewidth}{0.4pt} % Ajoute une ligne d'en-tête

% Configuration de l'en-tête pour les pages normales
\fancyhead[L]{\leftmark}
\fancyhead[R]{\rightmark}

% Configuration pour la première page
\fancypagestyle{plain}{
    \fancyhf{} % Efface les en-têtes et pieds de page par défaut
}

    
\begin{document} 

\newcommand{\anneeuniv}{2024--2025}
\newcommand{\master}{Master 1 TDSI - parcours Objets Connectés}
\newcommand{\titre}{Station Météo}
\renewcommand{\author}{GROUSSARD Tristan \& DJESSOU Koundeme Nobel \& MORET Maxime}
\newcommand{\teacher}{BOEGLEN Hervé \& RICHARD Noel}



\thispagestyle{empty}


\begin{titlepage}

\newgeometry{width=160mm,top=10mm,bottom=20mm}

  \begingroup

\begin{figure}[H]
  \centering
  \begin{minipage}{.5\textwidth}
    \flushleft
    \includegraphics[height=80pt]{figs/Logo-up.pdf}
  \end{minipage}%
  \begin{minipage}{.5\textwidth}
    \flushright
    \includegraphics[height=80pt]{figs/Logo-moc.png}
  \end{minipage}
\end{figure}

  
  \hrulefill

  \begin{center}
      \vspace{-0.5cm}
      Université de Poitiers, Année universitaire \anneeuniv\\
      U.F.R. Sciences Fondamentales Appliquées\\
      \master
  \end{center}
  
   \vspace{-3cm}

  \vfill
  
  \begin{center}
    {\huge\textbf{\titre}}\\
    \vspace{1.25cm}
    {\large Projet Module Systèmes Embarqués 2024}\\
    \vspace{3.5cm}
    \author\\
  \end{center}

  \vfill

  \hrulefill
  \flushright
  \vspace{-0.3cm}
  
  \ifdefempty{\teacher}{}{Module enseigné par \teacher\\}

  \endgroup

\restoregeometry
  
\end{titlepage}




\pagenumbering{gobble}


\tableofcontents
\label{toc} % Étiquette pour le lien


\listoffigures

\newpage


\clearpage
\setcounter{page}{3}
\pagenumbering{arabic}
\fancyfoot[C]{\hyperlink{toc}{\thepage} / \pageref{LastPage}} % Numéro de page au centre avec total
 %\fancyfoot[R]{\hyperlink{toc}{\thepage}}

\pagestyle{fancy}
\nsection{Introduction}

Dans le cadre de notre module de Systèmes Embarqués.
\begin{itemize}
    \item Réduire la dimensionnalité des données
    \item Mettre en évidence les structures et corrélations sous-jacentes
    \item Visualiser des informations complexes de manière synthétique
\end{itemize}

À travers ces exemples, nous explorerons les étapes clés de la méthode : prétraitement des données, calcul des composantes principales, projection et interprétation des résultats.


\vspace{0.5cm}


\newpage
\end{document}